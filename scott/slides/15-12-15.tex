\documentclass{beamer}
\usepackage{hyperref, mathrsfs, tikz, ulem}
\usecolortheme[RGB={34, 139, 34}]{structure}
\usetheme{Singapore}
\setbeamertemplate{navigation symbols}{\insertframenumber}

\def\A{\mathcal A}\def\B{\mathcal B}\def\C{\mathcal C}\def\D{\mathcal D}
\def\E{\mathcal E}\def\F{\mathcal F}\def\G{\mathcal G}\def\J{\mathcal J}
\def\L{\mathcal L}\def\M{\mathcal M}\def\N{\mathcal N}\def\O{\mathcal O}
\def\P{\mathcal P}\def\Q{\mathcal Q}\def\R{\mathcal R}\def\S{\mathcal S}
\def\T{\mathcal T}\def\W{\mathcal W}\def\V{\mathcal V}\def\X{\mathcal X}
\def\Y{\mathcal Y}\def\Z{\mathcal Z}
\def\mI{\mathbb I}\def\mN{\mathbb N}\def\mP{\mathbb P}\def\mR{\mathbb R}
\def\mS{\mathbb S}
\def\bA{{\bf A}}\def\bB{{\bf B}}\def\bC{{\bf C}}\def\bE{{\bf E}}\def\bP{{\bf P}}
\def\bU{{\bf U}}\def\bV{{\bf V}}\def\bX{{\bf X}}\def\bY{{\bf Y}}
\def\bx{{\bf x}}
\def\bSigma{{\bf \Sigma}}
\def\prob{\mbox{\bf prob}}\def\tr{\mbox{\bf tr}}
\def\l{\left}\def\r{\right}\def\lf{\lfloor}\def\rf{\rfloor}
\def\un{\underline}
\def\theat{\theta}\def\lambad{\lambda}\def\lamda{\lambda}
\def\iid{\stackrel{\mbox{\scriptsize i.i.d.}}{\sim}}
\def\ind{\stackrel{\mbox{\scriptsize ind.}}{\sim}}
\def\minimize{\mbox{minimize}\hspace{4mm}}
\def\maximize{\mbox{maximize}\hspace{4mm}}
\def\subjectto{\mbox{subject to}\hspace{4mm}}

\begin{document}

\begin{frame}{Introduction: sample problem (PAD)}
Peripheral Artery Disease (PAD) cohort:
\begin{itemize}
    \item $n = 5757$ patients
    \item Tx variable: cilostazol
    \item 20 outcomes measured
    \item explanatory variables:
    \begin{itemize}
        \item 6 demographic, 13 ``expert'' (used to determine Tx)
        \item 447 empirical variables
    \end{itemize}
\end{itemize}~\\
Goal: For a query patient, estimate which of treatment or control will be
more effective, i.e. yield a better outcome.\\
Beware observational data!
\end{frame}

\begin{frame}{Causal inference}
\begin{itemize}
\item Framework for using observational data when experimental data would be
    desired
\item Key assumption: {\bf given covariates}, treatment assignment is
    independent of (unobserved) response
\item General approach: create model to estimate propensity score, which is
    probability of treatment assignment
\item Takeaway from Low, Gallego \& Shaw (2015): PS methods struggled on
    intricate simulation
\end{itemize}
\end{frame}

\begin{frame}{A proposed approach}
Motivation: For a query patient, estimate which of treatment or control will be
more effective, i.e. yield a better outcome.\\
Beware observational data!
\begin{enumerate}
\item Estimate the propensity score for each patient (using e.g. random forest)
    and bin patients by propensity score.\\
    Within each bin:
\item Build a decision tree by, at each split, maximizing the difference
    between the two treatment effects estimated at the two daughter nodes.
\end{enumerate}
Can use important scores from both steps above to explain prediction to doctors
\end{frame}

\begin{frame}{Step 2 illustrated}
\end{frame}

\begin{frame}{Simulated data example}
Contingency table:
\begin{center}
\begin{tabular}{c|c|c|}
        & Control   & Treatment\\
        \hline
Male    & 10        & 50\\
\hline
Female  & 40        & 40\\
\hline
\end{tabular}\\~\\
$$Y_i = 10 + 10*\mI\{\mbox{sex} = F\} + 10*\mI\{\mbox{Tx} = 1\} + \epsilon_i$$
$$\epsilon_i \iid \mbox{Normal(0, \mbox{s.d.} = 5)}$$
\end{center}
Actual Tx effect: 10\\
Naive Tx effect estimate: 6
\end{frame}

\begin{frame}{Simulation results}
\centering
\includegraphics[height = \textheight]{../figs/simulation2.pdf}
\end{frame}

\begin{frame}{Next steps}
\begin{itemize}
\item Have framework for decision trees to identify heterogeneous treatment
    effects
    \begin{itemize}
    \item Can try making minor tweaks
    \item Can apply to more simulated datasets
    \end{itemize}
\end{itemize}
\end{frame}

\end{document}

